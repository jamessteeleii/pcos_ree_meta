% Options for packages loaded elsewhere
% Options for packages loaded elsewhere
\PassOptionsToPackage{unicode}{hyperref}
\PassOptionsToPackage{hyphens}{url}
\PassOptionsToPackage{dvipsnames,svgnames,x11names}{xcolor}
%
\documentclass[
]{article}
\usepackage{xcolor}
\usepackage[left=1in,right=1in,top=1in,bottom=1in]{geometry}
\usepackage{amsmath,amssymb}
\setcounter{secnumdepth}{-\maxdimen} % remove section numbering
\usepackage{iftex}
\ifPDFTeX
  \usepackage[T1]{fontenc}
  \usepackage[utf8]{inputenc}
  \usepackage{textcomp} % provide euro and other symbols
\else % if luatex or xetex
  \usepackage{unicode-math} % this also loads fontspec
  \defaultfontfeatures{Scale=MatchLowercase}
  \defaultfontfeatures[\rmfamily]{Ligatures=TeX,Scale=1}
\fi
\usepackage{lmodern}
\ifPDFTeX\else
  % xetex/luatex font selection
  \setmainfont[]{Latin Modern Roman}
\fi
% Use upquote if available, for straight quotes in verbatim environments
\IfFileExists{upquote.sty}{\usepackage{upquote}}{}
\IfFileExists{microtype.sty}{% use microtype if available
  \usepackage[]{microtype}
  \UseMicrotypeSet[protrusion]{basicmath} % disable protrusion for tt fonts
}{}
\makeatletter
\@ifundefined{KOMAClassName}{% if non-KOMA class
  \IfFileExists{parskip.sty}{%
    \usepackage{parskip}
  }{% else
    \setlength{\parindent}{0pt}
    \setlength{\parskip}{6pt plus 2pt minus 1pt}}
}{% if KOMA class
  \KOMAoptions{parskip=half}}
\makeatother
% Make \paragraph and \subparagraph free-standing
\makeatletter
\ifx\paragraph\undefined\else
  \let\oldparagraph\paragraph
  \renewcommand{\paragraph}{
    \@ifstar
      \xxxParagraphStar
      \xxxParagraphNoStar
  }
  \newcommand{\xxxParagraphStar}[1]{\oldparagraph*{#1}\mbox{}}
  \newcommand{\xxxParagraphNoStar}[1]{\oldparagraph{#1}\mbox{}}
\fi
\ifx\subparagraph\undefined\else
  \let\oldsubparagraph\subparagraph
  \renewcommand{\subparagraph}{
    \@ifstar
      \xxxSubParagraphStar
      \xxxSubParagraphNoStar
  }
  \newcommand{\xxxSubParagraphStar}[1]{\oldsubparagraph*{#1}\mbox{}}
  \newcommand{\xxxSubParagraphNoStar}[1]{\oldsubparagraph{#1}\mbox{}}
\fi
\makeatother


\usepackage{longtable,booktabs,array}
\usepackage{calc} % for calculating minipage widths
% Correct order of tables after \paragraph or \subparagraph
\usepackage{etoolbox}
\makeatletter
\patchcmd\longtable{\par}{\if@noskipsec\mbox{}\fi\par}{}{}
\makeatother
% Allow footnotes in longtable head/foot
\IfFileExists{footnotehyper.sty}{\usepackage{footnotehyper}}{\usepackage{footnote}}
\makesavenoteenv{longtable}
\usepackage{graphicx}
\makeatletter
\newsavebox\pandoc@box
\newcommand*\pandocbounded[1]{% scales image to fit in text height/width
  \sbox\pandoc@box{#1}%
  \Gscale@div\@tempa{\textheight}{\dimexpr\ht\pandoc@box+\dp\pandoc@box\relax}%
  \Gscale@div\@tempb{\linewidth}{\wd\pandoc@box}%
  \ifdim\@tempb\p@<\@tempa\p@\let\@tempa\@tempb\fi% select the smaller of both
  \ifdim\@tempa\p@<\p@\scalebox{\@tempa}{\usebox\pandoc@box}%
  \else\usebox{\pandoc@box}%
  \fi%
}
% Set default figure placement to htbp
\def\fps@figure{htbp}
\makeatother


% definitions for citeproc citations
\NewDocumentCommand\citeproctext{}{}
\NewDocumentCommand\citeproc{mm}{%
  \begingroup\def\citeproctext{#2}\cite{#1}\endgroup}
\makeatletter
 % allow citations to break across lines
 \let\@cite@ofmt\@firstofone
 % avoid brackets around text for \cite:
 \def\@biblabel#1{}
 \def\@cite#1#2{{#1\if@tempswa , #2\fi}}
\makeatother
\newlength{\cslhangindent}
\setlength{\cslhangindent}{1.5em}
\newlength{\csllabelwidth}
\setlength{\csllabelwidth}{3em}
\newenvironment{CSLReferences}[2] % #1 hanging-indent, #2 entry-spacing
 {\begin{list}{}{%
  \setlength{\itemindent}{0pt}
  \setlength{\leftmargin}{0pt}
  \setlength{\parsep}{0pt}
  % turn on hanging indent if param 1 is 1
  \ifodd #1
   \setlength{\leftmargin}{\cslhangindent}
   \setlength{\itemindent}{-1\cslhangindent}
  \fi
  % set entry spacing
  \setlength{\itemsep}{#2\baselineskip}}}
 {\end{list}}
\usepackage{calc}
\newcommand{\CSLBlock}[1]{\hfill\break\parbox[t]{\linewidth}{\strut\ignorespaces#1\strut}}
\newcommand{\CSLLeftMargin}[1]{\parbox[t]{\csllabelwidth}{\strut#1\strut}}
\newcommand{\CSLRightInline}[1]{\parbox[t]{\linewidth - \csllabelwidth}{\strut#1\strut}}
\newcommand{\CSLIndent}[1]{\hspace{\cslhangindent}#1}



\setlength{\emergencystretch}{3em} % prevent overfull lines

\providecommand{\tightlist}{%
  \setlength{\itemsep}{0pt}\setlength{\parskip}{0pt}}



 


\usepackage[font=scriptsize]{caption}
\usepackage[noblocks]{authblk}
\renewcommand*{\Authsep}{, }
\renewcommand*{\Authand}{, }
\renewcommand*{\Authands}{, }
\renewcommand\Affilfont{\small}
\usepackage{etoolbox}
\AtEndEnvironment{abstract}{\noindent\textbf{Keywords:} \KeywordList \par}
\makeatletter
\@ifpackageloaded{caption}{}{\usepackage{caption}}
\AtBeginDocument{%
\ifdefined\contentsname
  \renewcommand*\contentsname{Table of contents}
\else
  \newcommand\contentsname{Table of contents}
\fi
\ifdefined\listfigurename
  \renewcommand*\listfigurename{List of Figures}
\else
  \newcommand\listfigurename{List of Figures}
\fi
\ifdefined\listtablename
  \renewcommand*\listtablename{List of Tables}
\else
  \newcommand\listtablename{List of Tables}
\fi
\ifdefined\figurename
  \renewcommand*\figurename{Figure}
\else
  \newcommand\figurename{Figure}
\fi
\ifdefined\tablename
  \renewcommand*\tablename{Table}
\else
  \newcommand\tablename{Table}
\fi
}
\@ifpackageloaded{float}{}{\usepackage{float}}
\floatstyle{ruled}
\@ifundefined{c@chapter}{\newfloat{codelisting}{h}{lop}}{\newfloat{codelisting}{h}{lop}[chapter]}
\floatname{codelisting}{Listing}
\newcommand*\listoflistings{\listof{codelisting}{List of Listings}}
\makeatother
\makeatletter
\makeatother
\makeatletter
\@ifpackageloaded{caption}{}{\usepackage{caption}}
\@ifpackageloaded{subcaption}{}{\usepackage{subcaption}}
\makeatother
\usepackage{bookmark}
\IfFileExists{xurl.sty}{\usepackage{xurl}}{} % add URL line breaks if available
\urlstyle{same}
\hypersetup{
  pdftitle={Resting energy expenditure of women with and without polycystic ovary syndrome: a systematic review and meta-analysis},
  pdfauthor={Richard Kirwan; Leigh Peele; Gregory Nuckols; Georgia Kohlhoff; Hannah Cabré; Alyssa Olenick; James Steele},
  pdfkeywords={Polycystic ovary syndrome, Energy Metabolism, Resting
Metabolic Rate, Basal Metabolism},
  colorlinks=true,
  linkcolor={blue},
  filecolor={Maroon},
  citecolor={Blue},
  urlcolor={Blue},
  pdfcreator={LaTeX via pandoc}}


\title{Resting energy expenditure of women with and without polycystic
ovary syndrome: a systematic review and meta-analysis\thanks{Preprint,
please cite as: Kirwan, R., Peele, L., Nuckols, G., Kohlhoff, G., Cabré,
H., Olenick, A., and Steele, J. (2025). Resting energy expenditure of
women with and without polycystic ovary syndrome: a systematic review
and meta-analysis. medrxiv DOI: TO ADD. Address for correspondence:
james@steele-research.com}}


\author[1]{Richard Kirwan}
\author[2]{Leigh Peele}
\author[2]{Gregory Nuckols}
\author[3]{Georgia Kohlhoff}
\author[4]{Hannah Cabré}
\author[5]{Alyssa Olenick}
\author[2,6,7]{James Steele}

\affil[1]{Research Institute for Sports and Exercise Sciences, Liverpool
John Moores University, Liverpool, UK}
\affil[2]{MacroFactor, Stronger by Science Technologies LLC, Raleigh,
North Carolina, USA}
\affil[3]{Flourishing Health, Liverpool, UK}
\affil[4]{Reproductive Endocrinology and Women's Health Research
Program, Pennington Biomedical Research Center, Baton Rouge, Louisiana,
USA}
\affil[5]{Dr Alyssa Olenick LLC}
\affil[6]{Steele Research Limited, Eastleigh, Hampshire, UK}
\affil[7]{School of Health and Biomedical Sciences, Royal Melbourne
Institute of Technology, Melbourne, VIC, Australia}

% Store the keywords (expanded by Pandoc here)
\def\KeywordList{Polycystic ovary syndrome, Energy Metabolism, Resting
Metabolic Rate, Basal Metabolism}

\date{December 3, 2025}
\begin{document}
\maketitle
\begin{abstract}
Context: Polycystic ovary syndrome (PCOS) is common in reproductive-age
women, who often have higher BMI classification. This is assumed to stem
from lower resting energy expenditure (REE), influencing lifestyle
intervention guidelines. However, evidence for reduced REE in women with
PCOS compared with those without is inconsistent. Objective: To
systematically search and meta-analyse the existing literature to
estimate and describe the difference in REE between women with and
without PCOS. Data Sources: A systematic search was conducted using
PubMed, Medline and Web of Science databases of published research from
January 1990 to January 2025. Study Selection: Studies that measured REE
in women living with PCOS, both with and without control arms of women
without PCOS, were included. Data Extraction: Bibliometric, demographic,
and REE data was extracted by one investigator and checked in
triplicate. Data Synthesis: Thirteen studies were included in a Bayesian
arm-based multiple condition comparison (i.e., network) type
meta-analysis model with informative priors to compare both mean REE,
and between person variation in REE, between women with and without
PCOS. Mean REE differed between groups by 30 kcal/day {[}95\% quantile
interval: -47 to 113 kcal/day{]} and the contrast ratio for between
person standard deviations was 0.98 {[}95\% quantile interval: 0.71 to
1.33{]}. Conclusions: These findings indicate that REE does not
meaningfully differ between women with and without PCOS. Group-level
differences in resting energy expenditure are small, insignificant, or
not physiologically relevant.
\end{abstract}


\section{Introduction}\label{introduction}

Polycystic ovary syndrome (PCOS) affects approximately 10\% of women of
reproductive age worldwide, making it the most common endocrine disorder
affecting this population\textsuperscript{1}. Due to several factors
including hyperandrogenism and alterations in insulin resistance, PCOS
is believed to contribute to an increased risk of diabetes, metabolic
syndrome and cardiovascular disease\textsuperscript{2--6}, along with
being a leading cause of anovulatory infertility in
women\textsuperscript{7}. Furthermore, epidemiological data has
consistently demonstrated that women with PCOS are significantly more
likely to suffer from overweight or obesity, compared to the general
female population, with estimates ranging from 38\% to 88\% of PCOS
patients falling into overweight or obese body mass index (BMI)
categories\textsuperscript{8,9}.

The elevated incidence of overweight and obesity in PCOS is likely
multifactorial with proposed mechanisms including blunted postprandial
appetite hormone responses leading to reduced satiety and increased food
cravings\textsuperscript{10--12} and a reduced resting energy
expenditure (REE)\textsuperscript{13}. Indeed, the study from
Georgopoulos et al.\textsuperscript{13} examined REE in women with and
without PCOS using indirect calorimetry and reported that resting energy
expenditure was approximately \textasciitilde400 kcal/day lower in women
with PCOS. Notably, they also reported that insulin resistance further
reduced REE among women with PCOS, with insulin-resistant women
exhibiting an additional reduction nearly 500 kcal per day compared to
women with PCOS who were not insulin resistant\textsuperscript{13}.
Other studies similarly report lower REE in women with PCOS using
indirect methods (such as prediction from bioelectrical impedance
analysis or accelerometer physical activity data), also suggesting that
factors including insulin resistance and BMI category influence REE in
women with PCOS\textsuperscript{14,15}. However, despite the widespread
acceptance that women living with PCOS exhibit reduced REE based on
studies such as these, other research has reported little to no
difference in REE between women with and without
PCOS\textsuperscript{16--18}.

The consequences of widespread acceptance that REE is substantially
lower in women living with PCOS should not be underestimated,
particularly in light of the aforementioned incidence of overweight and
obesity in this population. Women with PCOS typically engage in more
frequent weight-loss attempts than women without
PCOS\textsuperscript{19}. From a physiological perspective, if women
with PCOS do exhibit a lower REE, this could imply a meaningful
metabolic disadvantage that may influence dietary and nutritional
guidance for weight management; for example, recommending a slightly
more severe energy restriction to overcome the belief that they have a
lower REE\textsuperscript{20}. Recommendations such as this could
influence the well documented prevalence of eating disorders in women
with PCOS\textsuperscript{21,22}. Contrastingly, belief in a ``slower
metabolism'' could instead serve as a deterrent to energy restriction
based weight-loss approaches for some women in line with typical general
population recommendations that are also recognised as efficacious for
improving PCOS symptoms\textsuperscript{23,24} and are routinely
recommended\textsuperscript{25}. It has been well documented that women
with PCOS already experience higher rates of anxiety, depression, and
lower quality of life (QOL) as a result of negative body image and
weight-related concerns\textsuperscript{26--29}. Clarifying the
relationship between REE and PCOS may therefore help guide more accurate
clinical recommendations and empower both practitioners and women with
PCOS.

Therefore, to estimate and describe the magnitude of difference in REE
between women with and without PCOS, we completed a systematic review
and meta-analysis of studies reporting REE in these populations.

\section{Materials and Methods}\label{materials-and-methods}

This systematic review and meta-analysis was pre-registered on PROSPERO
(\href{https://www.crd.york.ac.uk/PROSPERO/view/CRD42024601434}{CRD42024601434})
initially on the \(3^{rd}\) of December 2024 and performed in accordance
with the Preferred Reporting Items for Systematic Reviews and
Meta-Analyses (PRISMA) statement guidelines\textsuperscript{30}. The
PRISMA flow diagram reported below (Figure~\ref{fig-prisma}) was
produced using the \texttt{PRISMA2020} R package and Shiny
app\textsuperscript{31}. The primary aim of this review was to examine
the descriptive question ``Does resting energy expenditure (REE) differ
between women with and without polycystic ovary syndrome (PCOS)?''. We
summarise and describe the studies in addition to quantitatively
synthesising their results via meta-analysis.

\subsection{Search Strategy}\label{search-strategy}

PubMed, Web of Science, and MEDLINE databases were searched using the
following Boolean search string: ((``Basal Metabolic Rate''{[}MeSH{]} OR
``Energy Metabolism''{[}MeSH{]} OR ``Resting Metabolic Rate'' OR RMR OR
``Resting Energy Expenditure'' OR REE OR ``Basal Metabolic Rate'' OR BMR
OR ``resting energy'' OR ``basal energy expenditure'') AND (``Polycystic
Ovary Syndrome''{[}MeSH{]} OR ``Polycystic Ovary Syndrome'' OR PCOS OR
``Polycystic Ovarian Disease'' OR ``Stein-Leventhal Syndrome'')).
Searches were limited to publications up until May 2025 when the search
was completed, limited to English language articles, and Rayyan was used
to manage the search and screening process. Two reviewers (RK and GK)
independently screened all titles and abstracts against the predefined
inclusion and exclusion criteria. Articles deemed potentially eligible
by either reviewer were retrieved in full text. Full texts were then
independently assessed by RK and LP to determine final eligibility. Any
disagreements at either stage were resolved through discussion, and when
consensus could not be reached, a third reviewer acted as an
adjudicator.

\subsection{Eligibility Criteria}\label{eligibility-criteria}

Studies were included in the systematic review if 1) participants were
confirmed as women with PCOS between the ages of 18 to 65 years of age
with or without insulin resistance; 2) otherwise healthy (e.g.,
non-diabetic, no cardiovascular disease); 3) had a measure of REE
measured via multiple methods including direct/indirect calorimetry,
doubly labelled water; and 4) trials were not retracted at the time of
this analysis. Studies were excluded if they 1) used invalid or
non-standard methods for measuring REE (e.g., predicted REE from body
composition or accelerometer data); 2) non-peer-reviewed journal
articles (including grey literature sources such as conference
abstracts, theses and dissertations); and 3) were secondary analyses
with the same primary outcome data as another included study.

The condition being studied was PCOS and we included observational
cross-sectional design studies, in addition to intervention studies
where REE was reported for the population (and if present, the
comparator i.e., women without PCOS) condition of interest. For clarity,
studies of any design were included if they reported the REE using the
methods indicated for a sample of adult women with PCOS and who were
otherwise healthy. This included both studies with and without samples
of healthy control women without PCOS. As detailed in the statistical
analysis section below, a Bayesian model with informative priors based
on normative data for REE in healthy women without PCOS was included to
provide control information indirectly where this was missing. The use
of such priors is an efficient tool for incorporating historical
information on a particular population in a conservative
manner\textsuperscript{32}.

Following the PICO framework our eligibility criteria can be defined as
follows:

\begin{itemize}
\tightlist
\item
  Population

  \begin{itemize}
  \tightlist
  \item
    Inclusion criteria:

    \begin{itemize}
    \tightlist
    \item
      Women
    \item
      18-65 y
    \item
      With or without insulin resistance (IR)
    \end{itemize}
  \end{itemize}
\item
  Intervention(s) or exposure(s)

  \begin{itemize}
  \tightlist
  \item
    Otherwise healthy women with PCOS
  \end{itemize}
\item
  Comparator(s) or control(s)

  \begin{itemize}
  \tightlist
  \item
    Otherwise healthy control women without PCOS
  \end{itemize}
\item
  Outcome

  \begin{itemize}
  \tightlist
  \item
    Inclusion criteria

    \begin{itemize}
    \tightlist
    \item
      Resting energy expenditure (REE) measured via multiple methods
      including direct/indirect calorimetry, doubly labelled water.
    \end{itemize}
  \item
    Exclusion criteria:

    \begin{itemize}
    \tightlist
    \item
      Studies using invalid or non-standard methods for measuring REE
      (e.g., predicted REE from body composition or accelerometer data)
    \end{itemize}
  \end{itemize}
\end{itemize}

\subsection{Data extraction (selection and
coding)}\label{data-extraction-selection-and-coding}

Data was extracted by one investigator and checked in triplicate.
Bibliometric data including authors, journal, and article titles were
extracted. Descriptive statistics for age, body mass, fat mass, fat free
mass, height, BMI, race, physical activity levels, country of
investigation, information regarding glucose/insulin regulation and
insulin resistance status (where available), diagnostic criteria for
PCOS, and measurement method and device were extracted for each arm
within each study in addition to sample size. Descriptive
characteristics were then tabulated across studies for reporting.

For each arm, and observation time point if multiple observations
reported (e.g., before and after an intervention), depending on what was
reported by the authors we extracted the means, medians, standard
deviations, standard errors, lower and upper range values, and
interquartile range for the unadjusted and/or body mass adjusted and/or
fat free mass adjusted REE values. Where REE values adjusted for body
mass and/or fat free mass were reported we used the reported body mass
and/or fat free mass mean values for that arm to convert them to
unadjusted REE values (i.e., multiplied them by body mass and/or fat
free mass mean values). Where means and/or standard deviations were
missing the latter were either calculated from standard errors and
sample size, or all both were estimated from lower and upper range,
interquartile range, median, and sample size depending on the available
information using the methods of Wan et al.\textsuperscript{33}.
Further, where missing, height/body mass/BMI where estimated based on
the reported means. The units of measurement for which REE was extracted
and all REE values were converted to kcal/day. In one
case\textsuperscript{34} REE was reported relative to body mass and the
unadjusted values were no longer available (confirmed by the authors).
As such, in this case we used the mean body mass to convert back to
estimated REE unadjusted.

\subsubsection{Studies with possible reporting
errors}\label{studies-with-possible-reporting-errors}

During data extraction it was noted that several studies from the same
lab/research group\textsuperscript{13,35--37} contained a number of
discrepancies that seemed to be possible reporting errors. This
included, based on taking the authors results as written, standard
errors that implied impossible or at least incredibly unlikely standard
deviations, and discrepancies in sample size reporting throughout for
most variables without explanation or where this was explained the
sample sizes were discrepant with the text Further, data was not
reported for the healthy control women without PCOS in three of the
studies\textsuperscript{35--37}, and REE was reported as an ``adjusted''
value whereby
\(\textrm{REE}_{adjusted}=\textrm{REE}_{group~mean}+(\textrm{REE}_{adjusted}−\textrm{REE}_{predicted})\)
and the \(\textrm{REE}_{predicted}\) was obtained by substituting the
individual lean body mass, fat mass, gender, and age in the linear
regression equation generated by the data of all patients. In
correspondence with the senior author we were unable to clarify the
reporting discrepancies as the person responsible for the data/results
was no longer contactable. The original data were also no longer
available and so we could not calculated the unadjusted REE.

Given these issues we decided to extract the results from these studies
as reported and to conduct analyses both with and without their
inclusion. Though not pre-registered, due to a lack of confidence in the
reported results, we decided to include the analysis omitting these
studies as our main models in the results reported below. The results of
the analysis including them are reported in the sensitivity analysis
section.

\subsection{Statistical Analysis}\label{statistical-analysis}

Statistical analysis of the data extracted was be performed in R, (v
4.3.3; R Core Team, https://www.r-project.org/) and RStudio (v
2023.06.1; Posit, https://posit.co/). All code utilised for data
preparation, transformations, analyses, plotting, and reporting are
available in the corresponding GitHub repository
\url{https://github.com/jamessteeleii/pcos_ree_meta}. We cite all
software and packages used in the analysis pipeline using the
\texttt{grateful} package\textsuperscript{38} which can be seen here:
\url{https://github.com/jamessteeleii/pcos_ree_meta/blob/main/grateful-report.pdf}.
The statistical analysis plan was linked in our pre-registration
(PROSPERO:
\href{https://www.crd.york.ac.uk/PROSPERO/view/CRD42024601434}{CRD42024601434})
and available at the accompanying GitHub repository. Any deviations from
the pre-registration are noted below.

Given our research question our analysis was aimed at parameter
estimation\textsuperscript{39} within a Bayesian meta-analytic
framework\textsuperscript{40}. For all analyses model parameter
estimates and their precision, along with conclusions based upon them,
are interpreted continuously and probabilistically, considering data
quality, plausibility of effect, and previous literature, all within the
context of each model. The \texttt{renv} package\textsuperscript{41} was
used for package version reproducibility and a function based analysis
pipeline using the \texttt{targets} package\textsuperscript{42} was
employed (the analysis pipeline can be viewed by downloading the R
Project and running the function \texttt{targets::tar\_visnetwork()}).
Effect sizes and their variances were all calculated using the
\texttt{metafor} packages \texttt{escalc()}
function\textsuperscript{43}. The main package
\texttt{brms}\textsuperscript{44} was used in fitting all the Bayesian
meta-analysis models. Prior and posterior draws were taken using
\texttt{marginaleffects}\textsuperscript{45} and
\texttt{tidybayes}\textsuperscript{46} packages. All visualisations are
created using \texttt{ggplot2}\textsuperscript{47}, \texttt{tidybayes},
and the \texttt{patchwork}\textsuperscript{48} packages.

\subsubsection{Main Pre-registered
Models}\label{main-pre-registered-models}

We adopted an arm-based multiple condition comparison (i.e., network)
type model given that the studies included had arms of women with PCOS
both with, and without, a non-PCOS control arm\textsuperscript{49}, and
also in some cases multiple observations of REE in the different arms
included in the study (for example, where an intervention was conducted
and pre- and post-intervention REE was reported). In typical
contrast-based meta-analyses data is limited to the effect sizes for
paired contrasts between arms and thus studies that include both arms
(i.e., relative effects between non-PCOS control vs PCOS arms); however,
in arm-based analyses the data are the absolute effects within each arm
and information is borrowed across studies to enable both within
condition absolute, and between condition relative contrasts to be
estimated. We made use of historical information regarding REE in
healthy control women without PCOS by setting informative priors based
on meta-analysis of large scale studies reporting normative data for REE
in this population. This was included to provide indirect control
information where it was missing from particular studies. The use of
historical priors like this is an efficient tool to incorporate
historical information about a particular population in a conservative
manner in meta-analyses\textsuperscript{32}. From this model we focus on
reporting the global grand mean estimate for the fixed between condition
relative contrast for non-PCOS control vs PCOS arms as our primary
estimand of interest (i.e., \(\beta_1\) in both mean and standard
deviation models). We examined both raw mean REE (i.e., the absolute
mean REE in kcals per day for each arm) in addition to the between
person standard deviation in REE (i.e., the absolute standard deviation
in REE in kcals per day for each arm). Both models were multilevel in
that they included nested random intercepts for both study and arm
within study. In addition, and in deviation from our pre-registration,
we also included lab as a random intercept as in some cases we had
multiple studies from the same lab or research group. Lastly, the
inclusion of a random intercept for each effect size was accidentally
omitted from our pre-registration, and so this is also included in the
model.

\paragraph{Mean REE Model}\label{mean-ree-model}

The main model for mean REE with \(\textrm{cond}\) representing the
condition (either control or PCOS) was as follows: \[
\scriptsize
\begin{aligned}
\hat{\theta}_{ijkl} &\sim \mathcal{N}(\mu_{ijkl}, \sigma^2_{ijkl}) \\[6pt]
\mu_{ijkl} &= \beta_0 + \beta_1\,\text{cond}_[ijkl]
       + \alpha_{0,\mathrm{lab}[i]} + \alpha_{1,\mathrm{lab}[i]}\,\text{cond}
       + \alpha_{0,\mathrm{study}[j]} + \alpha_{1,\mathrm{study}[j]}\,\text{cond}
       + \alpha_{0,\mathrm{arm}[k]}
       + \alpha_{0,\mathrm{effect}[l]} \\[10pt]
\begin{pmatrix}
\alpha_{0,\mathrm{lab}[i]} \\[4pt]
\alpha_{1,\mathrm{lab}[i]}
\end{pmatrix}
&\sim
\mathcal{N}\!\left(
\begin{pmatrix} 0 \\[4pt] 0 \end{pmatrix},
\Sigma_{\mathrm{lab}}
\right),
\quad
\Sigma_{\mathrm{lab}} =
\begin{pmatrix}
\sigma^2_{0,\mathrm{lab}} & \rho_{\mathrm{lab}}\,\sigma_{0,\mathrm{lab}}\,\sigma_{1,\mathrm{lab}} \\[4pt]
\rho_{\mathrm{lab}}\,\sigma_{0,\mathrm{lab}}\,\sigma_{1,\mathrm{lab}} & \sigma^2_{1,\mathrm{lab}}
\end{pmatrix}
\text{, for lab i = 1,} \dots \text{,I} \\[12pt]
\begin{pmatrix}
\alpha_{0,\mathrm{study}[j]} \\[4pt]
\alpha_{1,\mathrm{study}[j]}
\end{pmatrix}
&\sim
\mathcal{N}\!\left(
\begin{pmatrix} 0 \\[4pt] 0 \end{pmatrix},
\Sigma_{\mathrm{study}}
\right),
\quad
\Sigma_{\mathrm{study}} =
\begin{pmatrix}
\sigma^2_{0,\mathrm{study}} & \rho_{\mathrm{study}}\,\sigma_{0,\mathrm{study}}\,\sigma_{1,\mathrm{study}} \\[4pt]
\rho_{\mathrm{study}}\,\sigma_{0,\mathrm{study}}\,\sigma_{1,\mathrm{study}} & \sigma^2_{1,\mathrm{study}}
\end{pmatrix}
\text{, for study j = 1,} \dots \text{,J} \\[12pt]
\alpha_{0,\mathrm{arm}[k]} &\sim \mathcal{N}(0, \sigma^2_{0,\mathrm{arm}})
\text{, for arm k = 1,} \dots \text{,K} \\[12pt]
\alpha_{0,\mathrm{effect}[l]} &\sim \mathcal{N}(0, \sigma^2_{0,\mathrm{effect}})
\text{, for effect l = 1,} \dots \text{,L}
\end{aligned}
\]

where \(\hat{\theta}_{ijkl}\) is the \(l\textrm{th}\) mean REE estimate
from the \(k\textrm{th}\) arm, for the \(j\textrm{th}\) study, conducted
by the \(i\textrm{th}\) lab and \(\sigma^2_{ijkl}\) is the corresponding
sampling error for that estimate. The random intercepts for the
\(i\textrm{th}\) lab, \(j\textrm{th}\) study, \(k\textrm{th}\) arm, and
\(l\textrm{th}\) mean REE estimate are \(\alpha_{0,lab[i]}\),
\(\alpha_{0,study[j]}\), \(\alpha_{0,arm[k]}\), and
\(\alpha_{0,effect[l]}\) respectively each with standard deviation of
\(\sigma^2_{0,lab[i]}\), \(\sigma^2_{0,study[j]}\),
\(\sigma^2_{0,arm[k]}\), and \(\sigma^2_{0,effect[l]}\). The parameter
\(\beta_0\) represents the fixed effect estimate of REE for control
conditions and \(\beta_1\) the fixed effect estimate for the offset from
this for the PCOS conditions (i.e., the difference between conditions).
The estimated offset was allowed to vary across both labs and studies
each reflected by \(\alpha_{1,lab[i]}\) and \(\alpha_{1,study[j]}\)
respectively, and these effects were also modelled as correlated with
the corresponding random intercepts with covariance \(\Sigma_{lab}\) and
\(\Sigma_{study}\), and \(\mathrm{corr}_{\mathrm{lab}}\) and
\(\mathrm{corr}_{\mathrm{study}}\) correlation matrices.

The priors for this model were as follows:

\[
\small
\begin{aligned}
\beta_0 &\sim \text{Student-}t(3,\, 1441.81,\, 84.56) \\[4pt]
\beta_1 &\sim \text{Student-}t(3,\, 0,\, 200) \\[12pt]
\sigma^2_{0,\mathrm{study}} &\sim \text{Half-student-}t(3,\, 149.89,\, 82.91) \\[4pt]
\sigma^2_{0,\mathrm{lab}},\, \sigma^2_{1,\mathrm{lab}},\,
\sigma^2_{1,\mathrm{study}},\, \sigma^2_{0,\mathrm{arm}}, \sigma^2_{0,\mathrm{effect}} &\sim \text{Half-student-}t(3,\, 0,\, 112.4) \\[12pt]
\mathrm{corr}_{\mathrm{lab}} &\sim \mathrm{LKJ}(1) \\[4pt]
\mathrm{corr}_{\mathrm{study}} &\sim \mathrm{LKJ}(1)
\end{aligned}
\] where the prior for \(\beta_{0}\), which corresponded to the model
intercept and mean REE in the control condition, was set based on
meta-analysis of the mean REEs for women from two large studies of
healthy people\textsuperscript{50,51} though set with a conservative
degrees of freedom for the \(\textrm{Student-}t\) distribution.The
random intercept \(\sigma^2_{0,study}\) was set similarly to this. The
prior for the fixed effect \(\beta_{1}\), reflecting the difference
between control and PCOS conditions was set based on a wide range of
possible values considering the minimum and and maximum values of the
ranges reported in the two studies noted (i.e., 2492 - 908 = 1584). We
then set a prior that permits values approximately across this range of
values with the majority of it's mass centred around zero. The remaining
random effects were set based on the default weakly regularising priors
for \texttt{brms} and scaled to the expected response values using a
\(\textrm{half-student-}t\) distribution with 3 degrees of freedom and
\(\mu=0\), and both correlation matrices \(\textrm{corr}_{lab}\) and
\(\textrm{corr}_{study}\) were set with an \(\textrm{LKJcorr(1)}\)
distribution.

\paragraph{Standard Deviation of REE
Model}\label{standard-deviation-of-ree-model}

The main model for the standard deviation of REE with \(\textrm{cond}\)
representing the condition (either control or PCOS) was as follows:

\[
\scriptsize
\begin{aligned}
\log(\hat{\theta}_{ijkl}) &\sim \mathcal{N}(\mu_{ijkl}, \sigma^2_{ijkl}) \\[6pt]
\mu_{ijkl} &= \beta_0 + \beta_1\,\text{cond}_{[ijkl]}
       + \beta_2\,\tilde{m}_{[ijkl]}
       + \alpha_{0,\mathrm{lab}[i]} + \alpha_{1,\mathrm{lab}[i]}\,\text{cond}
       + \alpha_{0,\mathrm{study}[j]} + \alpha_{1,\mathrm{study}[j]}\,\text{cond}
       + \alpha_{0,\mathrm{arm}[k]}
       + \alpha_{0,\mathrm{effect}[l]} \\[8pt]
m &= \log(y_{i,\mathrm{mean,[ijkl]}}) \\[3pt]
\tilde{m} &\sim \mathcal{N}(m,\, \sigma_{\log(y_{i,\mathrm{mean,[ijkl]}})}^2) \\[10pt]
\begin{pmatrix}
\alpha_{0,\mathrm{lab}[i]} \\[4pt]
\alpha_{1,\mathrm{lab}[i]}
\end{pmatrix}
&\sim
\mathcal{N}\!\left(
\begin{pmatrix} 0 \\[4pt] 0 \end{pmatrix},
\Sigma_{\mathrm{lab}}
\right),
\quad
\Sigma_{\mathrm{lab}} =
\begin{pmatrix}
\sigma^2_{0,\mathrm{lab}} & \rho_{\mathrm{lab}}\,\sigma_{0,\mathrm{lab}}\,\sigma_{1,\mathrm{lab}} \\[4pt]
\rho_{\mathrm{lab}}\,\sigma_{0,\mathrm{lab}}\,\sigma_{1,\mathrm{lab}} & \sigma^2_{1,\mathrm{lab}}
\end{pmatrix}
\text{, for lab i = 1,} \dots \text{,I} \\[12pt]
\begin{pmatrix}
\alpha_{0,\mathrm{study}[j]} \\[4pt]
\alpha_{1,\mathrm{study}[j]}
\end{pmatrix}
&\sim
\mathcal{N}\!\left(
\begin{pmatrix} 0 \\[4pt] 0 \end{pmatrix},
\Sigma_{\mathrm{study}}
\right),
\quad
\Sigma_{\mathrm{study}} =
\begin{pmatrix}
\sigma^2_{0,\mathrm{study}} & \rho_{\mathrm{study}}\,\sigma_{0,\mathrm{study}}\,\sigma_{1,\mathrm{study}} \\[4pt]
\rho_{\mathrm{study}}\,\sigma_{0,\mathrm{study}}\,\sigma_{1,\mathrm{study}} & \sigma^2_{1,\mathrm{study}}
\end{pmatrix}
\text{, for study j = 1,} \dots \text{,J} \\[12pt]
\alpha_{0,\mathrm{arm}[k]} &\sim \mathcal{N}(0, \sigma^2_{0,\mathrm{arm}})
\text{, for arm k = 1,} \dots \text{,K} \\[12pt]
\alpha_{0,\mathrm{effect}[l]} &\sim \mathcal{N}(0, \sigma^2_{0,\mathrm{effect}})
\text{, for effect l = 1,} \dots \text{,L}
\end{aligned}
\]

where \(\log(\hat{\theta}_{ijkl})\) is the \(l\textrm{th}\) natural
logarithm of the standard deviation of REE estimate from the
\(k\textrm{th}\) arm, for the \(j\textrm{th}\) study, conducted by the
\(i\textrm{th}\) lab and \(\sigma^2_{ijkl}\) is the corresponding
sampling error for that estimate. The random intercepts for the
\(i\textrm{th}\) lab, \(j\textrm{th}\) study, \(k\textrm{th}\) arm, and
\(l\textrm{th}\) standard deviation of REE estimate are
\(\alpha_{0,lab[i]}\), \(\alpha_{0,study[j]}\), \(\alpha_{0,arm[k]}\),
and \(\alpha_{0,effect[l]}\) respectively each with standard deviation
of \(\sigma^2_{0,lab[i]}\), \(\sigma^2_{0,study[j]}\),
\(\sigma^2_{0,arm[k]}\), and \(\sigma^2_{0,effect[l]}\). The parameter
\(\beta_0\) represents the fixed effect estimate of standard deviation
of REE for control conditions and \(\beta_1\) the fixed effect estimate
for the offset from this for the PCOS conditions (i.e., the difference
between conditions). The estimated offset was allowed to vary across
both labs and studies each reflected by \(\alpha_{1,lab[i]}\) and
\(\alpha_{1,study[j]}\) respectively, and these effects were also
modelled as correlated with the corresponding random intercepts with
covariance \(\Sigma_{lab}\) and \(\Sigma_{study}\), and
\(\mathrm{corr}_{\mathrm{lab}}\) and \(\mathrm{corr}_{\mathrm{study}}\)
correlation matrices. Finally, \(\beta_{2}\) represents the fixed effect
of the natural logarithm of the corresponding mean REE estimate
\(\tilde{m}\) which is modelled as estimated with measurement error
i.e., \(m\) represents the point estimate for the \(l\textrm{th}\)
natural logarithm of the mean REE estimate from the \(k\textrm{th}\)
arm, for the \(j\textrm{th}\) study, conducted by the \(i\textrm{th}\)
lab and \(\sigma_{\log(y_{i,\mathrm{mean,[ijkl]}})}^2\) is the
corresponding sampling error for that estimate.

The priors for this model were as follows: \[
\small
\begin{aligned}
\beta_0 &\sim \text{Student-}t(3,\, 5.54,\, 0.80) \\[4pt]
\beta_1 &\sim \text{Student-}t(3,\, 0,\, 5.3) \\[4pt]
\beta_2 &\sim \text{Student-}t(3,\, 0,\, 2.5) \\[12pt]
\text{mean}(\tilde{m}) &\sim \text{Half-student-}t(3,\, 7.28,\, 0.62) \\[4pt]
\text{sd}(\tilde{m}) &\sim \text{Half-student-}t(3,\, 0,\, 5) \\[12pt]
\sigma^2_{0,\mathrm{study}} &\sim \text{Half-student-}t(3,\, 1.08,\, 1.06) \\[4pt]
\sigma^2_{0,\mathrm{lab}},\, \sigma^2_{1,\mathrm{lab}},\,
\sigma^2_{1,\mathrm{study}},\, \sigma^2_{0,\mathrm{arm}}, \sigma^2_{0,\mathrm{effect}} &\sim \text{Half-student-}t(3,\, 0,\, 2.5) \\[12pt]
\mathrm{corr}_{\mathrm{lab}} &\sim \mathrm{LKJ}(1) \\[4pt]
\mathrm{corr}_{\mathrm{study}} &\sim \mathrm{LKJ}(1)
\end{aligned}
\]

where the prior for \(\beta_{0}\), which corresponded to the model
intercept and standard deviation of REE in the control condition, was
again set based on meta-analysis of the standard deviation of REEs for
women from two large studies of healthy people\textsuperscript{50,51}
though set with a conservative degrees of freedom for the
\(\textrm{Student-}t\) distribution.The random intercept
\(\sigma^2_{0,study}\) was set similarly to this. The prior for the
fixed effect \(\beta_{1}\), reflecting the difference between control
and PCOS conditions was set based on a wide range of possible values.
Given that in many cases of variables in the field there is an
approximate relationship of \textasciitilde1 for the natural logarithm
of the standard deviation conditioned upon the natural logarithm of the
mean\textsuperscript{52} we set this prior to reflect the range of
differences on the on the log scale (i.e., \(\log(1584)\)). We then set
a prior that permits values approximately across this range of values
with the majority of it's mass centred around zero. The prior for the
fixed effect \(\beta_{2}\), reflecting the relationship between the
natural logarithm of the mean REE with the natural logarithm of the
standard deviation of REE, was set it to be weakly informative centred
on zero with a wide scale to indicate uncertainty in this outcome
specifically despite the typical relationship close to \textasciitilde1.
Priors for the measurement error of the natural logarithm of the mean
REE estimate \(\tilde{m}\) were again based upon meta-analysis of the
aforementioned studies, though measurement error has to be positive, as
does the corresponding standard deviation of this error, so we set these
to conservative wide \(\textrm{half-student-}t\) distributions. The
remaining random effects were set based on the default weakly
regularising priors for \texttt{brms} and scaled to the expected
response values using a \(\textrm{half-student-}t\) distribution with 3
degrees of freedom and \(\mu=0\), and both correlation matrices
\(\textrm{corr}_{lab}\) and \(\textrm{corr}_{study}\) were set with an
\(\textrm{LKJcorr(1)}\) distribution.

\paragraph{Post-processing of models}\label{post-processing-of-models}

For both models we examined trace plots along with \(\hat{R}\) values to
examine whether chains have converged, and posterior predictive checks
for each model to understand the model implied distributions. From each
model we took draws from the posterior distributions for the conditional
absolute estimates for each condition (i.e., controls and PCOS) by study
incorporating random effects, the global grand mean absolute estimates
for each condition ignoring random effects, and the global grand mean
between condition relative contrast for controls vs PCOS conditions
ignoring random effects. The between condition relative contrast for
controls vs PCOS conditions corresponded to \(\beta{1}\) in each model
and was our primary estimand of interest; for the mean REE model this
corresponded to the absolute difference in mean REE, and for the
standard deviation of REE model this corresponded to the natural
logarithm of the ratio of standard deviations of REE which was
exponentiated (note, all log standard deviation of REE model estimates
were exponentiated back to the original scale to aid interpretibility).
We present the full probability density functions for posterior
visually, and also to calculate mean and 95\% quantile intervals (QI:
i.e., `credible' or `compatibility' intervals) for each estimate
providing the most probable value of the parameter in addition to the
range from 2.5\% to 97.5\% percentiles given our priors and data.

\subsubsection{Sensitivity analyses}\label{sensitivity-analyses}

\paragraph{Pairwise contrast based
models}\label{pairwise-contrast-based-models}

By way of pre-registered sensitivity analysis we also conducted pairwise
contrast based models where we limited the included effects to those
extracted from studies including only a directly comparable control and
PCOS arm at baseline. These models were both as follows:

\[
\small
\begin{aligned}
\hat{\theta}_{ij} &\sim \mathcal{N}(\mu_{ij}, \sigma^2_{ij}) \\[6pt]
\mu_{ij} &= \beta_0
       + \alpha_{0,\mathrm{lab}[i]}
       + \alpha_{0,\mathrm{study}[j]} \\[10pt]
\alpha_{0,\mathrm{lab}[i]} &\sim \mathcal{N}(0, \sigma^2_{0,\mathrm{lab}})
\text{, for lab i = 1,} \dots \text{,I} \\[12pt]
\alpha_{0,\mathrm{study}[j]} &\sim \mathcal{N}(0, \sigma^2_{0,\mathrm{study}})
\text{, for study j = 1,} \dots \text{,J}
\end{aligned}
\]

where \(\hat{\theta}_{ijkl}\) is the pairwise effect size, either the
mean difference in REE or the log coefficient of variation ratio
(calculated as PCOS vs control), for the \(j\textrm{th}\) study,
conducted by the \(i\textrm{th}\) lab and \(\sigma^2_{ijkl}\) is the
corresponding sampling error for that effect size estimate. The random
intercepts for the \(i\textrm{th}\) lab, \(j\textrm{th}\) study are
\(\alpha_{0,lab[i]}\) and \(\alpha_{0,study[j]}\) respectively each with
standard deviation of \(\sigma^2_{0,lab[i]}\),
\(\sigma^2_{0,study[j]}\). The parameter \(\beta_0\) represents the
fixed effect estimate of the pairwise effect size i.e., the pooled
estimate of the contrast between conditions. The priors for
\(\beta_{0}\) in these models were set as default weakly regularising
which is set on an intercept that results when internally centering all
population-level predictors around zero to improve sampling efficiency
and scaled to the expected response values using a
\(\textrm{student-}t\) distribution; for the mean difference in REE this
was \(\textrm{student-}t(3,\ 5.5,\ 50.2)\) and for the log coefficient
of variance ratio this was \(\textrm{student-}t(3,\ -0.1,\ 2.5)\). The
random effects were set similarly scaled to the expected response values
but using a \(\textrm{half-student-}t\) distribution centred on zero.
From these models we calculated the mean and 95\% quantile intervals
(i.e., `credible' or `compatibility' intervals) for the \(\beta_0\)
(comparable to the \(\beta_1\) from the corresponding mean and standard
deviation of REE arm-based models) for each effect size providing the
most probable value of the parameter in addition to the range from 2.5\%
to 97.5\% percentiles given our priors and data.

\paragraph{Additional sensitivity
analyses}\label{additional-sensitivity-analyses}

As noted in the section above, \emph{``Studies with possible reporting
errors''}, we opted to conduct analysis with and without the inclusion
of four studies with possible reporting errors we could not
resolve\textsuperscript{13,35--37}. Thus the main models described above
were run with and without these studies, the main results are presented
without them and the sensitivity results with them are reported
separately below.

As a further sensitivity analysis, given the inclusion of some studies
with interventions in women with PCOS reporting REE at multiple
timepoints such as mid- and post-intervention\textsuperscript{34,53,54}
but lacking of control women without PCOS, we opted to also conduct
sensitivity analysis excluding these and only examining their baseline
results in addition to the cross-sectional studies. As such, the main
models described above were run without the follow-up (i.e., mid- or
post-intervention timepoints) from these studies and only using the
baseline results in addition to other cross-sectional studies.

Lastly, given the role of body mass and, in particular, highly
metabolically active tissue on REE we included models adjusted for these
group level characteristics where studies reported them. These models
were essentially extensions of the main models noted above for BMI and
fat-free mass where each was modelled with it's corresponding sampling
error similarly to how the log mean REE was modelled in the models for
standard deviation of REE i.e., the mean BMI or fat-free mass estimates
were modelled as estimated with measurement error. When extracting
posterior distributions for the contrasts between conditions in these
models both BMI and fat-free mass where adjusted to the median values
seen in the control conditions i.e., BMI = 26.03 kg/\(\textrm{m}^2\) and
fat-free mass = 48.8 kg.

\section{Results}\label{results}

\subsection{Systematic Review}\label{systematic-review}

Note, the numbers in this section exclude the four studies previously
noted with possible reporting issues that were unresolved. These studies
are however summarised in fully in the descriptive characteristics table
in the online supplementary materials (see
\href{https://html-preview.github.io/?url=https://github.com/jamessteeleii/pcos_ree_meta/raw/refs/heads/main/tables/descriptives_table.html}{Descriptives
Table} where some of the possible reporting errors are seen in the
standard deviations reported or calculated e.g., the standard deviation
calculated for body mass of women with PCOS in Saltamavros et
al.\textsuperscript{36}. The PRISMA flow diagram
(Figure~\ref{fig-prisma}) also includes all studies identified including
the four with possible errors.

\begin{figure}

\centering{

\pandocbounded{\includegraphics[keepaspectratio]{prisma.png}}

}

\caption{\label{fig-prisma}PRISMA 2020 flow diagram template for
systematic reviews. Note that a ``report'' could be a journal article,
preprint, conference abstract, study register entry, clinical study
report, dissertation, unpublished manuscript, government report or any
other document providing relevant information. The website noted was a
prior narrative review on this topic by some of the authors
(https://macrofactorapp.com/pcos-bmr/) which identified two studies not
found in our systematic database search.}

\end{figure}%

Our systematic review identified 13 studies from 12 lab/research groups
including 24 arms (Control arms = 9, PCOS arms = 15) and a total of 918
participants (Controls: minimum n = 9, median n = 29, maximum n = 54;
PCOS: minimum n = 5, median n = 28, maximum n = 266). Descriptive
characteristics of the arms and participants in these studies are
reported fully in the online supplementary materials (see
\href{https://html-preview.github.io/?url=https://github.com/jamessteeleii/pcos_ree_meta/raw/refs/heads/main/tables/descriptives_table.html}{Descriptives
Table}).

Studies were carried out in multiple countries: Brazil (k = 3) ,and USA
(k = 3 studies), Australia, Cameroon, Canada, Italy, Sweden, Turkey, UK
(all k = 1 study). The four studies with noted reporting errors were
carried out in Greece. A total of 11 studies used the Rotterdam criteria
(or a modified version there of) for diagnosing
PCOS\textsuperscript{16,17,34,53--60}. One study used the 1990 National
Institutes of Health criteria\textsuperscript{61}, and two
studies\textsuperscript{18,55} diagnosed PCOS via the presence of
oligomenorrhea or amenorrhea alongside additional criteria including
plasma androgen levels, hirsutism or polycystic ovaries on ultrasound
scanning. All but one study\textsuperscript{61}, which used doubly
labelled water, measured REE using indirect calorimetry. The specific
devices reported by these studies are included in the online
supplementary materials (see
\href{https://html-preview.github.io/?url=https://github.com/jamessteeleii/pcos_ree_meta/raw/refs/heads/main/tables/descriptives_table.html}{Descriptives
Table}). We originally considered in our pre-registration that, given
sufficient data, we would compare sub groups of women with PCOS who did
and did not have accompanying insulin resistance. However, based on the
metabolic health variables reported in studies with this information
(see
\href{https://html-preview.github.io/?url=https://github.com/jamessteeleii/pcos_ree_meta/raw/refs/heads/main/tables/descriptives_table.html}{Descriptives
Table}) and, where available, considering primary criteria of either
homeostatic model assessment of insulin resistance (HOMA-IR) \(\geq2.5\)
or secondary criteria including fasting insulin \(>12\mu \textrm{U/mL}\)
or fasting glucose \(\geq100\textrm{mg/dL}\), all groups of women with
PCOS in the included studies would be considered to have insulin
resistance. Mean age of women with PCOS in these studies ranged from 23
to 33 which was similar to the control women without PCOS ranging from
23 to 30. Across those studies where BMI was reported or it was possible
to estimate the mean BMI of women with PCOS ranged from 26.4 to 39.9 was
typically greater than women without PCOS which ranged 20.5 to 27.9.

\subsection{Mean REE Model Results}\label{mean-ree-model-results}

The main model for mean REE resulted in a posterior distribution for the
contrast between control and PCOS conditions with a mean point estimate
of 30 kcal/day with a 95\% quantile interval ranging from -47 kcal/day
to 113 kcal/day suggesting there is a 95\% probability that the true
difference lies between these values given our priors and the data from
included studies. The corresponding conditional estimates for the
control condition and PCOS condition respectively were 1442 kcal/day
{[}95\%QI:1334 kcal/day to 1553 kcal/day{]} and 1472 kcal/day
{[}95\%QI:1359 kcal/day to 1587 kcal/day{]}. These results, including
the full visualisation of the posterior distribution and the conditional
estimates by study, can be seen in Figure~\ref{fig-mean-REE}.

\begin{figure}

\centering{

\pandocbounded{\includegraphics[keepaspectratio]{pre_print_files/figure-pdf/fig-mean-REE-1.pdf}}

}

\caption{\label{fig-mean-REE}Posterior distribution, mean point
estimates, and 95\% quantile intervals for conditional estimates by
study, global grand mean estimates by condition, and the contrast
between conditions for mean resting energy expenditure of control women
without PCOS and women with PCOS.}

\end{figure}%

\subsection{Standard Deviation of REE Model
Results}\label{standard-deviation-of-ree-model-results}

The main model for the between participant standard deviation of REE
resulted in a posterior distribution for the contrast ratio between
control and PCOS conditions with a mean point estimate of 0.98 with a
95\% quantile interval ranging from 0.71 to 1.33 suggesting there is a
95\% probability that the true ratio of standard deviations lies between
these values given our priors and the data from included studies. The
corresponding conditional estimates for the standard deviations of the
control condition and PCOS condition respectively were 238 kcal/day
{[}95\%QI:178 kcal/day to 312 kcal/day{]} and 229 kcal/day {[}95\%QI:169
kcal/day to 303 kcal/day{]}. These results, including the full
visualisation of the posterior distribution and the conditional
estimates by study, can be seen in Figure~\ref{fig-variance-REE}.

\begin{figure}

\centering{

\pandocbounded{\includegraphics[keepaspectratio]{pre_print_files/figure-pdf/fig-variance-REE-1.pdf}}

}

\caption{\label{fig-variance-REE}Posterior distribution, variance point
estimates, and 95\% quantile intervals for conditional estimates by
study, global grand variance estimates by condition, and the contrast
ratio between conditions for the standard deviation of resting energy
expenditure of control women without PCOS and women with PCOS.}

\end{figure}%

\subsection{Sensitivity Analyses}\label{sensitivity-analyses-1}

\subsubsection{Pairwise Models}\label{pairwise-models}

For mean REE the pairwise model resulted in qualitatively similar
inferences suggesting little difference between control and PCOS
conditions with mean point estimate of 16 kcal/day with a 95\% quantile
interval ranging from -36 kcal/day to 70 kcal/day. This was similar for
the standard deviation of REE with the pairwise model resulting in a
contrast ratio between the control and PCOS conditions with a mean point
estimate of 0.9 and a 95\% quantile interval ranging from 0.6 to 1.35.

\subsubsection{Models including studies with possible reporting
issues}\label{models-including-studies-with-possible-reporting-issues}

For mean REE the model including the studies noted with possible
reporting issues that were not resolved\textsuperscript{13,35--37} still
resulted in qualitatively similar inferences suggesting little
difference between control and PCOS conditions with mean point estimate
of 15 kcal/day with a 95\% quantile interval ranging from -69 kcal/day
to 95 kcal/day. This was similar for the standard deviation of REE with
the model including these studies resulting in a contrast ratio between
the control and PCOS conditions with a mean point estimate of 1.21 and a
95\% quantile interval ranging from 0.68 to 2.13.

\subsubsection{Basline REE Measurement
Models}\label{basline-ree-measurement-models}

For mean REE the model which included only baseline REE measurements
from studies involving interventions\textsuperscript{34,53,54}, in
addition to other cross-sectional studies, also resulted in
qualitatively similar inferences suggesting little difference between
control and PCOS conditions with mean point estimate of 32 kcal/day with
a 95\% quantile interval ranging from -46 kcal/day to 118 kcal/day. This
was also the case for the standard deviation of REE with this model
resulting in a contrast ratio between the control and PCOS conditions
with a mean point estimate of 0.97 and a 95\% quantile interval ranging
from 0.71 to 1.36.

\subsubsection{BMI and Fat-Free Mass Adjusted
Models}\label{bmi-and-fat-free-mass-adjusted-models}

For mean REE the model adjusted for BMI, also resulted in qualitatively
similar inferences suggesting little difference between control and PCOS
conditions with mean point estimate of 20 kcal/day with a 95\% quantile
interval ranging from -96 kcal/day to 140 kcal/day. This was also the
case for the standard deviation of REE with this model resulting in a
contrast ratio between the control and PCOS conditions with a mean point
estimate of 0.91 and a 95\% quantile interval ranging from 0.56 to 1.41.
This was similar for fat-free mass adjusted models too showing little
difference in mean REE between control and PCOS conditions with mean
point estimate of -9 kcal/day with a 95\% quantile interval ranging from
-195 kcal/day to 168 kcal/day and a contrast ratio between the control
and PCOS conditions with a mean point estimate of 0.79 and a 95\%
quantile interval ranging from 0.31 to 1.94.

\section{Discussion}\label{discussion}

This study sought to estimate and describe the magnitude of difference
in REE between women with and without PCOS. Most studies identified in
the systematic review, and included in the meta-analysis, used indirect
calorimetry as the primary measure of REE and assessed women with PCOS
who were insulin resistant and categorised as being in overweight or
obese BMI categories compared to healthy controls. Our results indicate
there is only a small magnitude of difference in REE (30 kcal/day
{[}95\%QI: -47 kcal/day to 113kcal/day{]}) between women with PCOS and
those without. Further, there is little difference in between person
variation between the groups based on the ratio of standard deviations
(0.98 {[}95\%QI: 0.71 to 1.33{]}) suggesting that, despite individual
differences in REE, PCOS is not systematically associated with lesser or
greater individual variability.

These findings challenge the widely held belief that PCOS is inherently
associated with a slower metabolism\textsuperscript{@ 25}, predisposing
women with PCOS to weight gain. This belief largely stems from a single
but influential 2009 study from Georgopoulos et al.~that reported a
significantly reduced BMR in women with PCOS\textsuperscript{13}, which
has been widely cited and reinforced in both academic and clinical
contexts. However, as we have noted, this study along with
others\textsuperscript{35--37} have numerous reporting errors which led
us to drop them from our present analysis (though sensitivity analysis
including them did not alter our overall conclusions). The mistaken
belief that REE is lower for those with PCOS may have mistakenly lead to
recommendations centred on slightly more severe calorie restriction to
achieve weight-loss goals, compared with recommendations for the general
population, as primary management strategies for women with
PCOS\textsuperscript{20}. Recognising that there may be minimal
differences in REE between women with PCOS and those without can inform
both clinical and public practices, potentially leading to a shift in
focus away from a requirement for more severe caloric restriction as a
primary method of treatment towards more comprehensive, individualised,
and psychologically safe approaches to care\textsuperscript{25}.

The current study suggests that REE may not a barrier to weight
regulation in PCOS given small group-level differences between women
with PCOS and those without (1472 kcal/day {[}95\%QI:1359 kcal/day to
1587 kcal/day{]} versus 1442 kcal/day {[}95\%QI:1334 kcal/day to 1553
kcal/day{]}, respectively). If anything, REE may be slightly higher in
women with PCOS compared to healthy women without PCOS. BMI of women
with PCOS ranged from 26.4 to 39.9 was typically greater than women
without PCOS which ranged 20.5 to 27.9 and this may explain the slightly
greater REE in the former group. However, for those studies where we
could extract or estimate BMI, our additional exploratory models
adjusted for this similarly showed little difference in REE (20 kcal/day
{[}95\%QI: -96 kcal/day to 140kcal/day{]}) between women with PCOS and
those without. However, one of the studies included in our
analysis\textsuperscript{62} reported that, whilst there was little
difference in unadjusted REE, when REE was adjusted for fat-free mass it
was lower in women with PCOS suggesting the potential importance of
fat-free tissue in energy regulation. Yet, for those studies where
fat-free mass was reported, our additional exploratory models adjusted
for this similarly showed little difference in REE (-9 kcal/day
{[}95\%QI: -195 kcal/day to 168kcal/day{]}) between women with PCOS and
those without. As such, even adjusted for both BMI and fat-free mass,
there seems to be little difference in REE between women with, and
without, PCOS. Yet, a recent systematic review of mechanisms for
metabolic dysfunction has reported excess androgen drives metabolic
issues within adipose tissue and muscle tissue contributing to
complications like obesity and insulin resistance\textsuperscript{63}.
Taken together, these findings highlight that factors beyond REE and
typical correlates of this including BMI or fat-free mass, such as
hormonal and tissue-specific metabolic effects, may play a more
significant role in weight regulation challenges in women with PCOS.

As noted, women with PCOS are more likely to engage in weight-loss
attempts\textsuperscript{19} and there could be concerns that this could
inadvertently further foster the already well documented disordered
eating in this population\textsuperscript{21,22,64,65}. The pathways
linking PCOS and disordered eating are multifactorial. Biological
mechanisms such as hyperandrogenism, hyperinsulinaemia, and altered
ghrelin and leptin signalling can heighten hunger, carbohydrate
cravings, and appetite variability\textsuperscript{66}. Frequent
hypoglycaemia and associated mood changes have also been observed, which
can trigger compensatory eating or binge episodes\textsuperscript{66}.
These physiological processes interact with psychological and social
stressors, including infertility concerns, conflicting nutrition advice,
chronic dieting, and exposure to idealised body images on social media,
which together compound vulnerability to disordered
eating\textsuperscript{67}. Moreover, eating disorders themselves can
disrupt endocrine function, potentially worsening PCOS symptoms and
creating a self-reinforcing cycle\textsuperscript{65,68}.

Understanding the intertwined biological, psychological, and social
influences suggests the importance of considering whether restrictive
dietary advice is appropriate given it may exacerbate feelings of
failure, hunger dysregulation, and shame\textsuperscript{69}. These
concerns, coupled with the lack of difference in REE between women with
and without PCOS might suggest that energy restriction based dietary
interventions for weight-loss may be unnecessary. But, there is also
evidence supporting the effect of energy restricted dietary
interventions for improving PCOS symptoms\textsuperscript{23,24} and
they are recommended in international guidelines\textsuperscript{25}.
Encouragingly though, these guidelines also recognise weight stigma as a
determinant of health and call for its reduction across clinical and
public health settings. Evidently the greater prevalence of women with
PCOS falling into overweight and obese BMI categories compared to women
without PCOS\textsuperscript{8,9} is unlikely to be due to differences
in REE and so, despite the potential effectiveness of energy restriction
dietary interventions, there is potential value in moving towards more
weight-neutral, individualised, and empowering care following holistic
guidelines recommending multiple approaches to
management\textsuperscript{20,25}.

\subsection{Strengths and Limitations}\label{strengths-and-limitations}

The current study has multiple strengths stemming from its
preregistered, comprehensive methodology and Bayesian statistical
approach. This statistical framework allowed us to incorporate studies
with and without control groups to better estimate REE in women with and
without PCOS and to perform multiple sensitivity analyses that confirmed
the stability of our findings. However, a limitation here is that
variability in methods across studies, such as differences in PCOS
diagnostic criteria and REE testing protocols, may have influenced
results and the relatively small number of studies overall limits the
extent to which we can explore these potential moderators. Furthermore,
some studies controlled for body weight or body composition when
reporting REE values, while others did not. We accounted for this by
estimating or converting reported data to obtain unadjusted REE values
across all groups, thereby reducing this variability and further as
noted above provided estimates adjusted for BMI and fat-free mass both
of which had little influence on our conclusions. Another limitation is
that, due to fewer total control groups than PCOS groups, informed
priors were required in several statistical models. However, in the
context of Bayesian meta-analysis this can also be considered a
strength. Additionally, some studies reported data inconsistencies that
could not be clarified (e.g.,\textsuperscript{13,35--37}) and were
dropped from our main analysis though our conclusions again did not
qualitative change when we conducted sensitivity analyses including
these studies. Finally, most included studies were cross-sectional or
baseline assessments within intervention trials, which limits causal
inference. Indeed, we did not pre-register any kind of causal model
(e.g., a directed acyclic graph) to inform our analysis approach for
causal inference and as such have been explicit about the estimates
presented as being descriptive.

\section{Conclusion}\label{conclusion}

In conclusion, the findings from this meta-analysis indicate that REE
does not meaningfully differ between women with and without PCOS.
Group-level differences in REE were small, insignificant, or not
physiologically relevant. Additionally, variability in REE between
individuals was also similar. These results suggest that a lower
baseline REE is not associated with the weight-related challenges often
associated with PCOS. These findings challenge the popular narrative
that women with PCOS have a lower REE and may help better inform dietary
interventions and nutritional support for these individuals. Future
research should include more standardized REE measurement and reporting
protocols, greater data transparency, consistent control and reporting
of body weight or body composition, the presentation of both absolute
and relative REE, and more precise characterization of PCOS phenotypes.
Overall, these findings support the conclusion that PCOS is not
negatively associated with REE and may help practitioners and
researchers focus on individually targeted and holistic lifestyle
interventions rather than negatively framed interventions based on
unsupported assumptions regarding REE.

\section{Financial Disclosures/Conflicts of
Interest}\label{financial-disclosuresconflicts-of-interest}

Richard Kirwan provides nutrition consultancy through his company Be
More Nutrition. Leigh Peele is employed by MacroFactor, a
nutrition-tracking app. Gregory Nuckols is co-owner of MacroFactor.
Georgia Kohlhoff provides nutrition consultancy through her company
Flourishing Health. Hannah Cabré was supported by the National Institute
of Diabetes and Digestive and Kidney Diseases of the National Institutes
of Health under Award Number T32DK064584. Alyssa Olenick provides
fitness coaching and scientific consulting under ALYSSA OLENICK LLC. She
receives compensation from scientific consulting, speaking engagments,
affiliated social media partnerships, and book publishing. James Steele
provides research consultancy through his company Steele Research
Limited, is contracted currently by MacroFactor and Kieser Australia
through Steele Research Limited, and has also received travel expenses
and honoraria for speaking from fit20 International, Exercise School
Portugal, and Discover Strength.

The content is solely the responsibility of the authors and does not
necessarily represent the official views of the National Institutes of
Health or any other funding agencies or institutions noted above. The
authors declare no other conflicts of interest related to the submitted
work.

\section{Data Availability}\label{data-availability}

All code utilised for data preparation, transformations, analyses,
plotting, and reporting are available in the corresponding GitHub
repository \url{https://github.com/jamessteeleii/pcos_ree_meta}.

\section{Contributions}\label{contributions}

Gregory Nuckols and Leigh Peele conceived the idea for the project. All
authors contributed to the design of the project and methods. Richie
Kirwan and Leigh Peele conducted the systematic search and screening.
James Steele performed the data extraction, conducted the statistical
analyses, and produced the data visualisations. All authors contributed
to drafting the initial manuscript. All authors contributed to editing
the manuscript. All authors read and approved the final manuscript.

\section*{References}\label{references}
\addcontentsline{toc}{section}{References}

\phantomsection\label{refs}
\begin{CSLReferences}{0}{1}
\bibitem[\citeproctext]{ref-salariGlobalPrevalencePolycystic2024}
\CSLLeftMargin{1. }%
\CSLRightInline{Salari N, Nankali A, Ghanbari A, et al. Global
prevalence of polycystic ovary syndrome in women worldwide: A
comprehensive systematic review and meta-analysis. \emph{Archives of
Gynecology and Obstetrics}. 2024;310(3):1303-1314.
doi:\href{https://doi.org/10.1007/s00404-024-07607-x}{10.1007/s00404-024-07607-x}}

\bibitem[\citeproctext]{ref-glintborgCardiovascularDiseaseNationwide2018}
\CSLLeftMargin{2. }%
\CSLRightInline{Glintborg D, Rubin KH, Nybo M, Abrahamsen B, Andersen M.
Cardiovascular disease in a nationwide population of {Danish} women with
polycystic ovary syndrome. \emph{Cardiovascular Diabetology}.
2018;17(1):37.
doi:\href{https://doi.org/10.1186/s12933-018-0680-5}{10.1186/s12933-018-0680-5}}

\bibitem[\citeproctext]{ref-glintborgProspectiveRiskType2022}
\CSLLeftMargin{3. }%
\CSLRightInline{Glintborg D, Kolster ND, Ravn P, Andersen MS.
Prospective {Risk} of {Type} 2 {Diabetes} in {Normal Weight Women} with
{Polycystic Ovary Syndrome}. \emph{Biomedicines}. 2022;10(6):1455.
doi:\href{https://doi.org/10.3390/biomedicines10061455}{10.3390/biomedicines10061455}}

\bibitem[\citeproctext]{ref-perssonHigherRiskType2021}
\CSLLeftMargin{4. }%
\CSLRightInline{Persson S, Elenis E, Turkmen S, Kramer MS, Yong EL,
Poromaa IS. Higher risk of type 2 diabetes in women with hyperandrogenic
polycystic ovary syndrome. \emph{Fertility and Sterility}.
2021;116(3):862-871.
doi:\href{https://doi.org/10.1016/j.fertnstert.2021.04.018}{10.1016/j.fertnstert.2021.04.018}}

\bibitem[\citeproctext]{ref-glintborgProspectiveRiskType2024}
\CSLLeftMargin{5. }%
\CSLRightInline{Glintborg D, Ollila MM, Møller JJK, et al. Prospective
risk of {Type} 2 diabetes in 99~892 {Nordic} women with polycystic ovary
syndrome and 446~055 controls: National cohort study from {Denmark},
{Finland}, and {Sweden}. \emph{Human Reproduction (Oxford, England)}.
2024;39(8):1823-1834.
doi:\href{https://doi.org/10.1093/humrep/deae124}{10.1093/humrep/deae124}}

\bibitem[\citeproctext]{ref-limMetabolicSyndromePolycystic2019}
\CSLLeftMargin{6. }%
\CSLRightInline{Lim SS, Kakoly NS, Tan JWJ, et al. Metabolic syndrome in
polycystic ovary syndrome: A systematic review, meta-analysis and
meta-regression. \emph{Obesity Reviews: An Official Journal of the
International Association for the Study of Obesity}. 2019;20(2):339-352.
doi:\href{https://doi.org/10.1111/obr.12762}{10.1111/obr.12762}}

\bibitem[\citeproctext]{ref-johamPrevalenceInfertilityUse2015}
\CSLLeftMargin{7. }%
\CSLRightInline{Joham AE, Teede HJ, Ranasinha S, Zoungas S, Boyle J.
Prevalence of infertility and use of fertility treatment in women with
polycystic ovary syndrome: Data from a large community-based cohort
study. \emph{Journal of Women's Health (2002)}. 2015;24(4):299-307.
doi:\href{https://doi.org/10.1089/jwh.2014.5000}{10.1089/jwh.2014.5000}}

\bibitem[\citeproctext]{ref-barberWhyAreWomen2022}
\CSLLeftMargin{8. }%
\CSLRightInline{Barber TM. Why are women with polycystic ovary syndrome
obese? \emph{British Medical Bulletin}. 2022;143(1):4-15.
doi:\href{https://doi.org/10.1093/bmb/ldac007}{10.1093/bmb/ldac007}}

\bibitem[\citeproctext]{ref-barberObesityPolycysticOvary2021}
\CSLLeftMargin{9. }%
\CSLRightInline{Barber TM, Franks S. Obesity and polycystic ovary
syndrome. \emph{Clinical Endocrinology}. 2021;95(4):531-541.
doi:\href{https://doi.org/10.1111/cen.14421}{10.1111/cen.14421}}

\bibitem[\citeproctext]{ref-hirschbergImpairedCholecystokininSecretion2004}
\CSLLeftMargin{10. }%
\CSLRightInline{Hirschberg AL, Naessén S, Stridsberg M, Byström B,
Holtet J. Impaired cholecystokinin secretion and disturbed appetite
regulation in women with polycystic ovary syndrome. \emph{Gynecological
Endocrinology: The Official Journal of the International Society of
Gynecological Endocrinology}. 2004;19(2):79-87.
doi:\href{https://doi.org/10.1080/09513590400002300}{10.1080/09513590400002300}}

\bibitem[\citeproctext]{ref-moranGhrelinMeasuresSatiety2004}
\CSLLeftMargin{11. }%
\CSLRightInline{Moran LJ, Noakes M, Clifton PM, et al. Ghrelin and
measures of satiety are altered in polycystic ovary syndrome but not
differentially affected by diet composition. \emph{The Journal of
Clinical Endocrinology and Metabolism}. 2004;89(7):3337-3344.
doi:\href{https://doi.org/10.1210/jc.2003-031583}{10.1210/jc.2003-031583}}

\bibitem[\citeproctext]{ref-stefanakiFoodCravingsObesity2024}
\CSLLeftMargin{12. }%
\CSLRightInline{Stefanaki K, Karagiannakis DS, Peppa M, et al. Food
{Cravings} and {Obesity} in {Women} with {Polycystic Ovary Syndrome}:
{Pathophysiological} and {Therapeutic Considerations}. \emph{Nutrients}.
2024;16(7):1049.
doi:\href{https://doi.org/10.3390/nu16071049}{10.3390/nu16071049}}

\bibitem[\citeproctext]{ref-georgopoulosBasalMetabolicRate2009}
\CSLLeftMargin{13. }%
\CSLRightInline{Georgopoulos NA, Saltamavros AD, Vervita V, et al. Basal
metabolic rate is decreased in women with polycystic ovary syndrome and
biochemical hyperandrogenemia and is associated with insulin resistance.
\emph{Fertility and Sterility}. 2009;92(1):250-255.
doi:\href{https://doi.org/10.1016/j.fertnstert.2008.04.067}{10.1016/j.fertnstert.2008.04.067}}

\bibitem[\citeproctext]{ref-romualdiRestingMetabolicRate2019}
\CSLLeftMargin{14. }%
\CSLRightInline{Romualdi D, Versace V, Tagliaferri V, et al. The resting
metabolic rate in women with polycystic ovary syndrome and its relation
to the hormonal milieu, insulin metabolism, and body fat distribution: A
cohort study. \emph{Journal of Endocrinological Investigation}.
2019;42(9):1089-1097.
doi:\href{https://doi.org/10.1007/s40618-019-01029-2}{10.1007/s40618-019-01029-2}}

\bibitem[\citeproctext]{ref-churchillBasalMetabolicRate2015}
\CSLLeftMargin{15. }%
\CSLRightInline{Churchill SJ, Wang ET, Bhasin G, et al. Basal metabolic
rate in women with {PCOS} compared to eumenorrheic controls.
\emph{Clinical Endocrinology}. 2015;83(3):384-388.
doi:\href{https://doi.org/10.1111/cen.12740}{10.1111/cen.12740}}

\bibitem[\citeproctext]{ref-graffSaturatedFatIntake2017}
\CSLLeftMargin{16. }%
\CSLRightInline{Graff SK, Mario FM, Magalhães JA, Moraes RS, Spritzer
PM. Saturated {Fat Intake Is Related} to {Heart Rate Variability} in
{Women} with {Polycystic Ovary~Syndrome}. \emph{Annals of Nutrition \&
Metabolism}. 2017;71(3-4):224-233.
doi:\href{https://doi.org/10.1159/000484325}{10.1159/000484325}}

\bibitem[\citeproctext]{ref-larssonDietaryIntakeResting2016}
\CSLLeftMargin{17. }%
\CSLRightInline{Larsson I, Hulthén L, Landén M, Pålsson E, Janson P,
Stener-Victorin E. Dietary intake, resting energy expenditure, and
eating behavior in women with and without polycystic ovary syndrome.
\emph{Clinical Nutrition (Edinburgh, Scotland)}. 2016;35(1):213-218.
doi:\href{https://doi.org/10.1016/j.clnu.2015.02.006}{10.1016/j.clnu.2015.02.006}}

\bibitem[\citeproctext]{ref-segalRestingMetabolicRate1990}
\CSLLeftMargin{18. }%
\CSLRightInline{Segal KR, Dunaif A.
\href{https://www.ncbi.nlm.nih.gov/pubmed/2228390}{Resting metabolic
rate and postprandial thermogenesis in polycystic ovarian syndrome}.
\emph{International Journal of Obesity}. 1990;14(7):559-567.}

\bibitem[\citeproctext]{ref-pesonenPolycysticOvarySyndrome2023}
\CSLLeftMargin{19. }%
\CSLRightInline{Pesonen E, Nurkkala M, Niemelä M, et al. Polycystic
ovary syndrome is associated with weight-loss attempts and perception of
overweight independent of {BMI}: A population-based cohort study.
\emph{Obesity (Silver Spring, Md)}. 2023;31(4):1108-1120.
doi:\href{https://doi.org/10.1002/oby.23681}{10.1002/oby.23681}}

\bibitem[\citeproctext]{ref-ozgensaydamWeightManagementStrategies2021}
\CSLLeftMargin{20. }%
\CSLRightInline{Ozgen Saydam B, Yildiz BO. Weight management strategies
for patients with {PCOS}: Current perspectives. \emph{Expert Review of
Endocrinology \& Metabolism}. 2021;16(2):49-62.
doi:\href{https://doi.org/10.1080/17446651.2021.1896966}{10.1080/17446651.2021.1896966}}

\bibitem[\citeproctext]{ref-lalonde-besterPrevalenceEtiologyEating2024}
\CSLLeftMargin{21. }%
\CSLRightInline{Lalonde-Bester S, Malik M, Masoumi R, et al. Prevalence
and {Etiology} of {Eating Disorders} in {Polycystic Ovary Syndrome}: {A
Scoping Review}. \emph{Advances in Nutrition (Bethesda, Md)}.
2024;15(4):100193.
doi:\href{https://doi.org/10.1016/j.advnut.2024.100193}{10.1016/j.advnut.2024.100193}}

\bibitem[\citeproctext]{ref-jeanesBingeEatingBehaviours2017}
\CSLLeftMargin{22. }%
\CSLRightInline{Jeanes YM, Reeves S, Gibson EL, Piggott C, May VA, Hart
KH. Binge eating behaviours and food cravings in women with {Polycystic
Ovary Syndrome}. \emph{Appetite}. 2017;109:24-32.
doi:\href{https://doi.org/10.1016/j.appet.2016.11.010}{10.1016/j.appet.2016.11.010}}

\bibitem[\citeproctext]{ref-aleneziImpactDietInducedWeight2024}
\CSLLeftMargin{23. }%
\CSLRightInline{Alenezi SA, Elkmeshi N, Alanazi A, Alanazi ST, Khan R,
Amer S. The {Impact} of {Diet-Induced Weight Loss} on {Inflammatory
Status} and {Hyperandrogenism} in {Women} with {Polycystic Ovarian
Syndrome} ({PCOS})-{A Systematic Review} and {Meta-Analysis}.
\emph{Journal of Clinical Medicine}. 2024;13(16):4934.
doi:\href{https://doi.org/10.3390/jcm13164934}{10.3390/jcm13164934}}

\bibitem[\citeproctext]{ref-holteRestoredInsulinSensitivity1995}
\CSLLeftMargin{24. }%
\CSLRightInline{Holte J, Bergh T, Berne C, Wide L, Lithell H. Restored
insulin sensitivity but persistently increased early insulin secretion
after weight loss in obese women with polycystic ovary syndrome.
\emph{The Journal of Clinical Endocrinology and Metabolism}.
1995;80(9):2586-2593.
doi:\href{https://doi.org/10.1210/jcem.80.9.7673399}{10.1210/jcem.80.9.7673399}}

\bibitem[\citeproctext]{ref-teedeRecommendations2023International2023}
\CSLLeftMargin{25. }%
\CSLRightInline{Teede HJ, Tay CT, Laven JJE, et al. Recommendations
{From} the 2023 {International Evidence-based Guideline} for the
{Assessment} and {Management} of {Polycystic Ovary Syndrome}. \emph{The
Journal of Clinical Endocrinology and Metabolism}.
2023;108(10):2447-2469.
doi:\href{https://doi.org/10.1210/clinem/dgad463}{10.1210/clinem/dgad463}}

\bibitem[\citeproctext]{ref-himeleinDepressionBodyImage2006}
\CSLLeftMargin{26. }%
\CSLRightInline{Himelein MJ, Thatcher SS. Depression and body image
among women with polycystic ovary syndrome. \emph{Journal of Health
Psychology}. 2006;11(4):613-625.
doi:\href{https://doi.org/10.1177/1359105306065021}{10.1177/1359105306065021}}

\bibitem[\citeproctext]{ref-hofmannBodyImageMental2025}
\CSLLeftMargin{27. }%
\CSLRightInline{Hofmann K, Decrinis C, Bitterlich N, Tropschuh K, Stute
P, Bachmann A. Body image and mental health in women with polycystic
ovary syndrome-a cross-sectional study. \emph{Archives of Gynecology and
Obstetrics}. 2025;312(1):177-190.
doi:\href{https://doi.org/10.1007/s00404-024-07913-4}{10.1007/s00404-024-07913-4}}

\bibitem[\citeproctext]{ref-gellerBodyImageIllness2025}
\CSLLeftMargin{28. }%
\CSLRightInline{Geller S, Levy S, Avitsur R. Body image, illness
perception, and psychological distress in women coping with polycystic
ovary syndrome ({PCOS}). \emph{Health Psychology Open}.
2025;12:20551029251327441.
doi:\href{https://doi.org/10.1177/20551029251327441}{10.1177/20551029251327441}}

\bibitem[\citeproctext]{ref-davitadzeBodyImageConcerns2023}
\CSLLeftMargin{29. }%
\CSLRightInline{Davitadze M, Malhotra K, Khalil H, et al. Body image
concerns in women with polycystic ovary syndrome: A systematic review
and meta-analysis. \emph{European Journal of Endocrinology}.
2023;189(2):R1-R9.
doi:\href{https://doi.org/10.1093/ejendo/lvad110}{10.1093/ejendo/lvad110}}

\bibitem[\citeproctext]{ref-pagePRISMA2020Statement2021a}
\CSLLeftMargin{30. }%
\CSLRightInline{Page MJ, McKenzie JE, Bossuyt PM, et al. The {PRISMA}
2020 statement: An updated guideline for reporting systematic reviews.
\emph{BMJ}. 2021;372:n71.
doi:\href{https://doi.org/10.1136/bmj.n71}{10.1136/bmj.n71}}

\bibitem[\citeproctext]{ref-haddawayPRISMA2020PackageShiny2022}
\CSLLeftMargin{31. }%
\CSLRightInline{Haddaway NR, Page MJ, Pritchard CC, McGuinness LA.
{PRISMA2020}: {An R} package and {Shiny} app for producing {PRISMA}
2020-compliant flow diagrams, with interactivity for optimised digital
transparency and {Open Synthesis}. \emph{Campbell Systematic Reviews}.
2022;18(2):e1230.
doi:\href{https://doi.org/10.1002/cl2.1230}{10.1002/cl2.1230}}

\bibitem[\citeproctext]{ref-weberApplyingMetaAnalyticPredictivePriors2021}
\CSLLeftMargin{32. }%
\CSLRightInline{Weber S, Li Y, Iii JWS, Kakizume T, Schmidli H. Applying
{Meta-Analytic-Predictive Priors} with the {R Bayesian Evidence
Synthesis Tools}. \emph{Journal of Statistical Software}. 2021;100:1-32.
doi:\href{https://doi.org/10.18637/jss.v100.i19}{10.18637/jss.v100.i19}}

\bibitem[\citeproctext]{ref-wanEstimatingSampleMean2014}
\CSLLeftMargin{33. }%
\CSLRightInline{Wan X, Wang W, Liu J, Tong T. Estimating the sample mean
and standard deviation from the sample size, median, range and/or
interquartile range. \emph{BMC Medical Research Methodology}.
2014;14(1):135.
doi:\href{https://doi.org/10.1186/1471-2288-14-135}{10.1186/1471-2288-14-135}}

\bibitem[\citeproctext]{ref-pohlmeierEffectLowstarchLowdairy2014}
\CSLLeftMargin{34. }%
\CSLRightInline{Pohlmeier AM, Phy JL, Watkins P, et al. Effect of a
low-starch/low-dairy diet on fat oxidation in overweight and obese women
with polycystic ovary syndrome. \emph{Applied Physiology, Nutrition, and
Metabolism = Physiologie Appliquee, Nutrition Et Metabolisme}.
2014;39(11):1237-1244.
doi:\href{https://doi.org/10.1139/apnm-2014-0073}{10.1139/apnm-2014-0073}}

\bibitem[\citeproctext]{ref-kritikouA2BV3Adrenergic2006}
\CSLLeftMargin{35. }%
\CSLRightInline{Kritikou S, Saltamavros AD, Adonakis G, et al. The
{\(\alpha\)2B} and {\(\beta\)}3 adrenergic receptor genes polymorphisms
in women with polycystic ovarian syndrome ({PCOS}) and their association
with insulin resistance and basal metabolic rate ({BMR}). \emph{Review
of Clinical Pharmacology and Pharmacokinetics -- International Edition}.
2006;2006(2).}

\bibitem[\citeproctext]{ref-saltamavrosAlpha2Beta2007}
\CSLLeftMargin{36. }%
\CSLRightInline{Saltamavros AD, Adonakis G, Kritikou S, et al. Alpha 2
beta adrenoreceptor 301-303 deletion polymorphism in polycystic ovary
syndrome. \emph{Clinical Autonomic Research: Official Journal of the
Clinical Autonomic Research Society}. 2007;17(2):112-114.
doi:\href{https://doi.org/10.1007/s10286-007-0403-6}{10.1007/s10286-007-0403-6}}

\bibitem[\citeproctext]{ref-koikaAssociationPro12AlaPolymorphism2009}
\CSLLeftMargin{37. }%
\CSLRightInline{Koika V, Marioli DJ, Saltamavros AD, et al. Association
of the {Pro12Ala} polymorphism in peroxisome proliferator-activated
receptor Gamma2 with decreased basic metabolic rate in women with
polycystic ovary syndrome. \emph{European Journal of Endocrinology}.
2009;161(2):317-322.
doi:\href{https://doi.org/10.1530/EJE-08-1014}{10.1530/EJE-08-1014}}

\bibitem[\citeproctext]{ref-rodriguez-sanchezGratefulFacilitateCitation2023}
\CSLLeftMargin{38. }%
\CSLRightInline{Rodriguez-Sanchez F, cre, cph, Jackson CP, Hutchins SD,
Clawson JM. Grateful: {Facilitate Citation} of {R Packages}. Published
online October 2023.}

\bibitem[\citeproctext]{ref-cummingNewStatisticsWhy2014}
\CSLLeftMargin{39. }%
\CSLRightInline{Cumming G. The {New Statistics}: {Why} and {How}.
\emph{Psychological Science}. 2014;25(1):7-29.
doi:\href{https://doi.org/10.1177/0956797613504966}{10.1177/0956797613504966}}

\bibitem[\citeproctext]{ref-kruschkeBayesianNewStatistics2018}
\CSLLeftMargin{40. }%
\CSLRightInline{Kruschke JK, Liddell TM. The {Bayesian New Statistics}:
{Hypothesis} testing, estimation, meta-analysis, and power analysis from
a {Bayesian} perspective. \emph{Psychonomic Bulletin \& Review}.
2018;25:178-206.}

\bibitem[\citeproctext]{ref-usheyRenvProjectEnvironments2023}
\CSLLeftMargin{41. }%
\CSLRightInline{Ushey K, cre, Wickham H, Software P, PBC. Renv: {Project
Environments}. Published online September 2023.}

\bibitem[\citeproctext]{ref-landauTargetsDynamicFunctionOriented2023}
\CSLLeftMargin{42. }%
\CSLRightInline{Landau WM, Warkentin MT, Edmondson M, Oliver S, Mahr T,
Company EL and. Targets: {Dynamic Function-Oriented} '{Make}'-{Like
Declarative Pipelines}. Published online October 2023.}

\bibitem[\citeproctext]{ref-viechtbauerMetaforMetaAnalysisPackage2023}
\CSLLeftMargin{43. }%
\CSLRightInline{Viechtbauer W. Metafor: {Meta-Analysis Package} for {R}.
Published online September 2023.}

\bibitem[\citeproctext]{ref-burknerBrmsBayesianRegression2023}
\CSLLeftMargin{44. }%
\CSLRightInline{Bürkner PC, Gabry J, Weber S, et al. Brms: {Bayesian
Regression Models} using '{Stan}'. Published online September 2023.}

\bibitem[\citeproctext]{ref-arel-bundockMarginaleffectsPredictionsComparisons2023}
\CSLLeftMargin{45. }%
\CSLRightInline{Arel-Bundock V, cre, cph, Diniz MA, Greifer N, Bacher E.
Marginaleffects: {Predictions}, {Comparisons}, {Slopes}, {Marginal
Means}, and {Hypothesis Tests}. Published online October 2023.}

\bibitem[\citeproctext]{ref-kayTidybayesTidyData2023}
\CSLLeftMargin{46. }%
\CSLRightInline{Kay M, Mastny T. Tidybayes: {Tidy Data} and '{Geoms}'
for {Bayesian Models}. Published online August 2023.}

\bibitem[\citeproctext]{ref-wickhamGgplot2CreateElegant2023}
\CSLLeftMargin{47. }%
\CSLRightInline{Wickham H, Chang W, Henry L, et al. Ggplot2: {Create
Elegant Data Visualisations Using} the {Grammar} of {Graphics}.
Published online October 2023.}

\bibitem[\citeproctext]{ref-pedersenPatchworkComposerPlots2023}
\CSLLeftMargin{48. }%
\CSLRightInline{Pedersen TL. Patchwork: {The Composer} of {Plots}.
Published online August 2023.}

\bibitem[\citeproctext]{ref-hongBayesianMissingData2016}
\CSLLeftMargin{49. }%
\CSLRightInline{Hong H, Chu H, Zhang J, Carlin BP. A {Bayesian} missing
data framework for generalized multiple outcome mixed treatment
comparisons. \emph{Research Synthesis Methods}. 2016;7(1):6-22.
doi:\href{https://doi.org/10.1002/jrsm.1153}{10.1002/jrsm.1153}}

\bibitem[\citeproctext]{ref-velasquezUseAmmoniaInhalants2011}
\CSLLeftMargin{50. }%
\CSLRightInline{Velasquez JR. The {Use} of {Ammonia Inhalants Among
Athletes}. \emph{Strength \& Conditioning Journal}. 2011;33(2):33.
doi:\href{https://doi.org/10.1519/SSC.0b013e3181fd5c9b}{10.1519/SSC.0b013e3181fd5c9b}}

\bibitem[\citeproctext]{ref-pavlidouRevisedHarrisBenedict2023}
\CSLLeftMargin{51. }%
\CSLRightInline{Pavlidou E, Papadopoulou SK, Seroglou K, Giaginis C.
Revised {Harris}--{Benedict Equation}: {New Human Resting Metabolic Rate
Equation}. \emph{Metabolites}. 2023;13(2):189.
doi:\href{https://doi.org/10.3390/metabo13020189}{10.3390/metabo13020189}}

\bibitem[\citeproctext]{ref-steeleMetaanalysisVariationSport2023a}
\CSLLeftMargin{52. }%
\CSLRightInline{Steele J, Fisher, Smith, Schoenfeld, Yang, and Nakagawa
S. Meta-analysis of variation in sport and exercise science: {Examples}
of application within resistance training research. \emph{Journal of
Sports Sciences}. 2023;41(17):1617-1634.
doi:\href{https://doi.org/10.1080/02640414.2023.2286748}{10.1080/02640414.2023.2286748}}

\bibitem[\citeproctext]{ref-moranShorttermMealReplacements2006}
\CSLLeftMargin{53. }%
\CSLRightInline{Moran LJ, Noakes M, Clifton PM, Wittert GA, Williams G,
Norman RJ. Short-term meal replacements followed by dietary
macronutrient restriction enhance weight loss in polycystic ovary
syndrome. \emph{The American Journal of Clinical Nutrition}.
2006;84(1):77-87.
doi:\href{https://doi.org/10.1093/ajcn/84.1.77}{10.1093/ajcn/84.1.77}}

\bibitem[\citeproctext]{ref-brunerEffectsExerciseNutritional2006}
\CSLLeftMargin{54. }%
\CSLRightInline{Bruner B, Chad K, Chizen D. Effects of exercise and
nutritional counseling in women with polycystic ovary syndrome.
\emph{Applied Physiology, Nutrition, and Metabolism = Physiologie
Appliquee, Nutrition Et Metabolisme}. 2006;31(4):384-391.
doi:\href{https://doi.org/10.1139/h06-007}{10.1139/h06-007}}

\bibitem[\citeproctext]{ref-robinsonPostprandialThermogenesisReduced1992}
\CSLLeftMargin{55. }%
\CSLRightInline{Robinson S, Chan SP, Spacey S, Anyaoku V, Johnston DG,
Franks S. Postprandial thermogenesis is reduced in polycystic ovary
syndrome and is associated with increased insulin resistance.
\emph{Clinical Endocrinology}. 1992;36(6):537-543.
doi:\href{https://doi.org/10.1111/j.1365-2265.1992.tb02262.x}{10.1111/j.1365-2265.1992.tb02262.x}}

\bibitem[\citeproctext]{ref-cosarRestingMetabolicRate2008}
\CSLLeftMargin{56. }%
\CSLRightInline{Cosar E, Köken G, Sahin FK, et al. Resting metabolic
rate and exercise capacity in women with polycystic ovary syndrome.
\emph{International Journal of Gynaecology and Obstetrics: The Official
Organ of the International Federation of Gynaecology and Obstetrics}.
2008;101(1):31-34.
doi:\href{https://doi.org/10.1016/j.ijgo.2007.10.011}{10.1016/j.ijgo.2007.10.011}}

\bibitem[\citeproctext]{ref-graffDietaryGlycemicIndex2013}
\CSLLeftMargin{57. }%
\CSLRightInline{Graff SK, Mário FM, Alves BC, Spritzer PM. Dietary
glycemic index is associated with less favorable anthropometric and
metabolic profiles in polycystic ovary syndrome women with different
phenotypes. \emph{Fertility and Sterility}. 2013;100(4):1081-1088.
doi:\href{https://doi.org/10.1016/j.fertnstert.2013.06.005}{10.1016/j.fertnstert.2013.06.005}}

\bibitem[\citeproctext]{ref-dohRelationshipAdiposityInsulin2016}
\CSLLeftMargin{58. }%
\CSLRightInline{Doh E, Mbanya A, Kemfang-Ngowa JD, et al. The
{Relationship} between {Adiposity} and {Insulin Sensitivity} in {African
Women Living} with the {Polycystic Ovarian Syndrome}: {A Clamp Study}.
\emph{International Journal of Endocrinology}. 2016;2016:9201701.
doi:\href{https://doi.org/10.1155/2016/9201701}{10.1155/2016/9201701}}

\bibitem[\citeproctext]{ref-rodriguesLowValidityPredictive2018}
\CSLLeftMargin{59. }%
\CSLRightInline{Rodrigues AM dos S, Costa ABP, Campos DL, et al. Low
validity of predictive equations for calculating resting energy
expenditure in overweight and obese women with polycystic ovary
syndrome. \emph{Journal of Human Nutrition and Dietetics}.
2018;31(2):266-275.
doi:\href{https://doi.org/10.1111/jhn.12498}{10.1111/jhn.12498}}

\bibitem[\citeproctext]{ref-tosiInsulinMediatedSubstrateUse2021}
\CSLLeftMargin{60. }%
\CSLRightInline{Tosi F, Villani M, Migazzi M, et al. Insulin-{Mediated
Substrate Use} in {Women With Different Phenotypes} of {PCOS}: The
{Role} of {Androgens}. \emph{The Journal of Clinical Endocrinology and
Metabolism}. 2021;106(9):e3414-e3425.
doi:\href{https://doi.org/10.1210/clinem/dgab380}{10.1210/clinem/dgab380}}

\bibitem[\citeproctext]{ref-broskeyAssessingEnergyRequirements2017}
\CSLLeftMargin{61. }%
\CSLRightInline{Broskey NT, Klempel MC, Gilmore LA, et al. Assessing
{Energy Requirements} in {Women With Polycystic Ovary Syndrome}: {A
Comparison Against Doubly Labeled Water}. \emph{The Journal of Clinical
Endocrinology and Metabolism}. 2017;102(6):1951-1959.
doi:\href{https://doi.org/10.1210/jc.2017-00459}{10.1210/jc.2017-00459}}

\bibitem[\citeproctext]{ref-tosiRestingEnergyExpenditure2024}
\CSLLeftMargin{62. }%
\CSLRightInline{Tosi F, Rosmini F, Gremes V, et al. Resting energy
expenditure in women with polycystic ovary syndrome. \emph{Human
Reproduction (Oxford, England)}. 2024;39(8):1794-1803.
doi:\href{https://doi.org/10.1093/humrep/deae129}{10.1093/humrep/deae129}}

\bibitem[\citeproctext]{ref-sanchez-garridoMetabolicDysfunctionPolycystic2020}
\CSLLeftMargin{63. }%
\CSLRightInline{Sanchez-Garrido MA, Tena-Sempere M. Metabolic
dysfunction in polycystic ovary syndrome: {Pathogenic} role of androgen
excess and potential therapeutic strategies. \emph{Molecular
Metabolism}. 2020;35:100937.
doi:\href{https://doi.org/10.1016/j.molmet.2020.01.001}{10.1016/j.molmet.2020.01.001}}

\bibitem[\citeproctext]{ref-leeIncreasedRiskDisordered2017}
\CSLLeftMargin{64. }%
\CSLRightInline{Lee I, Cooney LG, Saini S, et al. Increased risk of
disordered eating in~polycystic ovary syndrome. \emph{Fertility and
Sterility}. 2017;107(3):796-802.
doi:\href{https://doi.org/10.1016/j.fertnstert.2016.12.014}{10.1016/j.fertnstert.2016.12.014}}

\bibitem[\citeproctext]{ref-bernadettPrevalenceEatingDisorders2016}
\CSLLeftMargin{65. }%
\CSLRightInline{Bernadett M, Szemán-N A.
\href{https://www.ncbi.nlm.nih.gov/pubmed/27244869}{{{[}Prevalence of
eating disorders among women with polycystic ovary syndrome{]}}}.
\emph{Psychiatria Hungarica: A Magyar Pszichiatriai Tarsasag Tudomanyos
Folyoirata}. 2016;31(2):136-145.}

\bibitem[\citeproctext]{ref-barryImpactEatingBehavior2011}
\CSLLeftMargin{66. }%
\CSLRightInline{Barry JA, Bouloux P, Hardiman PJ. The impact of eating
behavior on psychological symptoms typical of reactive hypoglycemia. {A}
pilot study comparing women with polycystic ovary syndrome to controls.
\emph{Appetite}. 2011;57(1):73-76.
doi:\href{https://doi.org/10.1016/j.appet.2011.03.003}{10.1016/j.appet.2011.03.003}}

\bibitem[\citeproctext]{ref-steegers-theunissenPolycysticOvarySyndrome2020}
\CSLLeftMargin{67. }%
\CSLRightInline{Steegers-Theunissen RPM, Wiegel RE, Jansen PW, Laven
JSE, Sinclair KD. Polycystic {Ovary Syndrome}: {A Brain Disorder
Characterized} by {Eating Problems Originating} during {Puberty} and
{Adolescence}. \emph{International Journal of Molecular Sciences}.
2020;21(21):8211.
doi:\href{https://doi.org/10.3390/ijms21218211}{10.3390/ijms21218211}}

\bibitem[\citeproctext]{ref-cooneyIncreasedPrevalenceBinge2024}
\CSLLeftMargin{68. }%
\CSLRightInline{Cooney LG, Gyorfi K, Sanneh A, et al. Increased
{Prevalence} of {Binge Eating Disorder} and {Bulimia Nervosa} in {Women
With Polycystic Ovary Syndrome}: {A Systematic Review} and
{Meta-Analysis}. \emph{The Journal of Clinical Endocrinology and
Metabolism}. 2024;109(12):3293-3305.
doi:\href{https://doi.org/10.1210/clinem/dgae462}{10.1210/clinem/dgae462}}

\bibitem[\citeproctext]{ref-pirottaDisorderedEatingBehaviours2019}
\CSLLeftMargin{69. }%
\CSLRightInline{Pirotta S, Barillaro M, Brennan L, et al. Disordered
{Eating Behaviours} and {Eating Disorders} in {Women} in {Australia}
with and without {Polycystic Ovary Syndrome}: {A Cross-Sectional Study}.
\emph{Journal of Clinical Medicine}. 2019;8(10):1682.
doi:\href{https://doi.org/10.3390/jcm8101682}{10.3390/jcm8101682}}

\end{CSLReferences}




\end{document}
